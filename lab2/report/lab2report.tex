\documentclass[a4paper,12pt]{article}

\usepackage[hidelinks]{hyperref}
\usepackage{amsmath}
\usepackage{mathtools}
\usepackage{shorttoc}
\usepackage{cmap}
\usepackage[T2A]{fontenc}
\usepackage[utf8]{inputenc}
\usepackage[english, russian]{babel}
\usepackage{xcolor}
\usepackage{graphicx}
\usepackage{float}
\graphicspath{{./img/}}

\definecolor{linkcolor}{HTML}{000000}
\definecolor{urlcolor}{HTML}{0085FF}
\hypersetup{pdfstartview=FitH,
            linkcolor=linkcolor,
            urlcolor=urlcolor,
            colorlinks=true}

\DeclarePairedDelimiter{\floor}{\lfloor}{\rfloor}

\renewcommand*\contentsname{Содержание}

\newcommand{\plot}[3]{
    \begin{figure}[H]
        \begin{center}
            \includegraphics[scale=0.6]{#1}
            \caption{#2}
            \label{#3}
        \end{center}
    \end{figure}
}

\begin{document}
    \include{title}
    \newpage

    \tableofcontents
    \listoffigures
    \newpage

    \section{Постановка задачи}
    \quad Имеется выборка $ (X, (Y)) $. $ X $ -- множество вещественных чисел,
    $ Y $ -- множество интервалов. Необходимо восстановить функциональную зависимость.
    
    \section{Теория}
    \quad Для выборки $ (X, (Y))$, $ X = \{x_i\}_{i=1}^{n}, {Y} = \{{y}_i\}_{i=1}^{n} $
    ($ x_i $ - точеный, $ {y}_i $ - интервальный) линейная регрессионная модель имеет вид:

    \begin{equation}
        y = \beta_0 + \beta_1 x
        \label{e:model}
    \end{equation}

    Для оценки параметров необходимо решить систему вида:

    \begin{equation}
        \begin{gathered}
        \underline{y_i} \leq  y = \beta_0 + \beta_1 x_i \leq \overline{y_i} \\
        i = 1..n
        \end{gathered}
    \end{equation}

    С учетом применения метода вариации неопределенности имеем задачу минимизации:

    \begin{equation}
        \begin{gathered}
            \sum_{i = 1}^{n}w_{i} \to \min \\
            \text{mid}{y}_{i} - w_{i} \text{rad}{y}_{i} \leq \beta_0 + \beta_1 x_i \leq \text{mid}{y}_{i} + w_{i} \text{rad}{y}_{i} \\
            w_{i} \geq 0, i = 1..n \\
        \end{gathered}
    \end{equation}

    \quad Информационным множеством называется множество всех значений параметров
    $ \beta_0, \beta_1 $, удовлетворяющих \ref{e:model}. Минимальные и максимальные значения
    параметров в информационном множестве определяют внешнюю оценку параметров модели.

    \quad Коридором совместных зависимостей называется множетсво всех модельных функций
    совместных с исходными данными.

    \section{Реализация}
    \quad Весь код написан на языке Python (версии 3.9.5).
    \href{https://github.com/BoIlAl/Intervals/tree/master/lab2}{Ссылка на GitHub с исходным кодом}.

    \section{Результаты}
    \quad Данные были взяты из файлов \textsl{-0\_25V/-0\_25V\_13.txt}, \textsl{-0\_5V/-0\_5V\_13.txt}, \textsl{+0\_25V/+0\_25V\_13.txt} и
    \textsl{+0\_5V/+0\_5V\_13.txt}. С коррекцией при помощи вспомогательных данных из 
    файла \textsl{ZeroLine/ZeroLine\_13.txt}. Набор значений
    $ X = [-0.5, -0.25, 0.25, 0.5] $. Набор значений $ Y_1 $ определяется как интервальная
    мода данных из соответсвующих файлов (изначальные данные обыинтерваливаются с 
    $eps = 2^{-5} $). Набор значений $ Y_2 $ определяется как обынтерваленное среднее 
    из соответсвующих файлов ($eps = 2^{-5} $). 
    
    Начнем с $ Y_1 $. Итоговая выборка:
    \plot{img/X, (Y1).png}{Исходная интервальная выборка $ X, (Y_1) $}{p:y1}
    
    Точечная линейная регрессия имеет вид: 
    \plot{img/Regression X, (Y1).png}{Точечная линейная регрессия выборки $ X, (Y_1) $}{p:regY1}

    Точечные оценки параметров: $ \beta_0 = 0.0, \beta_1 = 0.86236 $.

    Построим информационное множество:
    \plot{img/Inform X, (Y1).png}{Информационное множество выборки $ X, (Y_1) $}{p:infY1}

    Интервальные оценки параметров: $ \beta_0 = [-0.01635, 0.00856],
    \beta_1 = [0.82965, 0.89135] $.

    Коридор совместных зависимостей:
    \plot{img/Corridor X, (Y1).png}{Коридор совместных зависимостей выборки $ X, (Y_1) $}{p:corY1}

    Теперь $ Y_2 $. Итоговая выборка:
    \plot{img/X, (Y2).png}{Исходная интервальная выборка $ X, (Y_2) $}{p:y2}

    Точечная линейная регрессия имеет вид: 
    \plot{img/Regression X, (Y2).png}{Точечная линейная регрессия выборки $ X, (Y_2) $}{p:regY2}
    
    Точечные оценки параметров: $ \beta_0 = 0.0003, \beta_1 = 0.85377 $.

    Построим информационное множество:
    \plot{img/Inform X, (Y2).png}{Информационное множество выборки $ X, (Y_2) $}{p:infY2}
    
    Интервальные оценки параметров: $ \beta_0 = [-0.02984, 0.03146],
    \beta_1 = [0.78907, 0.91407] $.

    Коридор совместных зависимостей:
    \plot{img/Corridor X, (Y2).png}{Коридор совместных зависимостей выборки $ X, (Y_2) $}{p:corY2}
    
    \section{Обсуждение}
    \quad Из полученых результатов можно заметить, что оценки выборки $ X, (Y_1) $ имеют примерно вдвое меньшую неопределенность, чем выборки $ X, (Y_2) $. Для обеих выборок точечные оценки параметров модели лежат внутри информационного множества, и как следствие, линия регрессии лежит внутри коридора совместных зависимостей.
    
\end{document}