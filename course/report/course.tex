\documentclass[a4paper,12pt]{article}

\usepackage[hidelinks]{hyperref}
\usepackage{amsmath}
\usepackage{mathtools}
\usepackage{shorttoc}
\usepackage{cmap}
\usepackage[T2A]{fontenc}
\usepackage[utf8]{inputenc}
\usepackage[english, russian]{babel}
\usepackage{xcolor}
\usepackage{graphicx}
\usepackage{float}
\graphicspath{{./img/}}

\definecolor{linkcolor}{HTML}{000000}
\definecolor{urlcolor}{HTML}{0085FF}
\hypersetup{pdfstartview=FitH,
            linkcolor=linkcolor,
            urlcolor=urlcolor,
            colorlinks=true}

\DeclarePairedDelimiter{\floor}{\lfloor}{\rfloor}

\renewcommand*\contentsname{Содержание}

\newcommand{\plot}[3]{
    \begin{figure}[H]
        \begin{center}
            \includegraphics[scale=0.6]{#1}
            \caption{#2}
            \label{#3}
        \end{center}
    \end{figure}
}

\begin{document}
    \begin{titlepage}
	\begin{center}
		{\large Санкт-Петербургский политехнический университет\\Петра Великого\\}
	\end{center}
	
	\begin{center}
		{\large Физико-механический иститут}
	\end{center}
	
	
	\begin{center}
		{\large Кафедра «Прикладная математика»}
	\end{center}
	
	\vspace{8em}
	
	\begin{center}
		{\bfseries Отчёт по лабораторной работе №2 \\по дисциплине «Анализ данных с интервальной неопределённостью»}
	\end{center}
	
	\vspace{5em}
	
	\begin{flushleft}
		\hspace{16em}Выполнил студент:\\\hspace{16em}Бочкарев Илья Алексеевич\\\hspace{16em}группа: 5040102/20201
		
		\vspace{2em}
		
		\hspace{16em}Проверил:\\\hspace{16em}к.ф.-м.н., доцент\\\hspace{16em}Баженов Александр Николаевич
		
	\end{flushleft}
	
	
	\vspace{6em}
	
	
	\begin{center}
		Санкт-Петербург\\2023 г.
	\end{center}	
	
\end{titlepage}
    \newpage

    \tableofcontents
    \listoffigures
    \newpage

    \section{Постановка задачи}
    \quad Решить СЛАУ, правая часть которой представлена в виде трапециевидных нечетких чисел, а левая -- матрицей томографии. Сравнить различные способы решения.
    
    \section{Теория}
    \quad Имеется фиксированное количество областей, находящихся на определенном расстоянии от камеры-обскура, за которой стоят датчики. Геометрическая модель такой конструкции представляет собой $ m $ окружностей (областей) и $ n $ хорд \ref{p:model}. Элементы матрицы томографии находятся как точки пересечения хорд с каждой из окружностей.

    \plot{img/tom.png}{Геометрическая модель}{p:model}

    \quad Имеется СЛАУ, левая часть которой представлена матрицей томографии, а правая представляет собой вектор трапециевидных нечетких чисел:

    \begin{equation}
        Ax = fuzzy(a, b, \gamma, \delta)
        \label{e:slau}
    \end{equation}

    Рассмотрим два способа решения СЛАУ такого вида:

    Независмо решить СЛАУ стандартными методами для ядра $ [a, b] $ и для носителя $ [\gamma, \delta] $ с использованием интервальной арифметики. Из полученных результатов сконструировать вектор-решение трапециевидных нечетких чисел.

    Решить СЛАУ стандартными методами с использованием арифметики нечетких чисел.

    Для поиска точечного решения системы независимо для ядра $ [a, b] $ и для носителя $ [\gamma, \delta] $ решается оптимизационная задача вида:

    \begin{equation}
        \begin{gathered}
            \sum_{i = 1}^{n}w_{i} \to \min \\
            \text{mid}\underline{y_i} - w_{i} \text{rad}\underline{y_i} \leq \sum_{j = 1}^{m}{a_{ij}x_j} \leq \text{mid}\overline{y_i} + w_{i} \text{rad}\overline{y_i} \\
            w_{i} \geq 0, i = 1..n, \\
        \end{gathered}
        \label{e:point}
    \end{equation}

    \section{Реализация}
    \quad Весь код написан на языке Python (версии 3.9.5).
    \href{https://github.com/BoIlAl/Intervals/tree/master/course}{Ссылка на GitHub с исходным кодом}.

    \section{Результаты}
    \quad Рассматривается модель с $ m = n = 4 $. Точное решение $ x_j = 1, j = 1..m $, отсюда находится  $ b $. Вектор нечетких чисел: $ bf_j = fuzzy(b_j - \epsilon, b_j + \epsilon, b_j - 2 * \epsilon, b_j + 2 * \epsilon) $, где $ \epsilon = 0.0001 $. В качестве метода решения СЛАУ использован матричный.

    \quad Полученный вектор правой части имеет вид:

    \begin{equation}
        \begin{gathered}
            bf = [[1.1814, 1.1816], [1.1813, 1.1817]], \\
            [[1.19049, 1.19069], [1.19039, 1.19079]],  \\
            [[1.19652, 1.19672], [1.19642, 1.19682]], \\
            [[1.19952, 1.19972], [1.19942, 1.19982]] \\
        \end{gathered}
    \end{equation}
    
    \quad Точечное решение:
    \begin{equation}
        x = 0.9999999997, 0.9999999998, 0.999999997, 1.000000003
    \end{equation}

    \quad Результат раздельного решения:

    \begin{equation}
        \begin{gathered}
            x = [[-3.37019, 5.37019], [-7.74037, 9.74037]], \\
            [[-1.56107, 3.56107], [-4.12214, 6.12214]],  \\
            [[-42.63182, 44.63182], [-86.26364, 88.26364]], \\
            [[-49.56259, 51.56259], [-100.12518, 102.12518]] \\
        \end{gathered}
    \end{equation}

    \quad Решение с использованием арифметики нечетких чисел:

   \begin{equation}
        \begin{gathered}
            x = [[-3.37019, 5.37019], [-7.74037, 9.74037]], \\
            [[-1.56107, 3.56107], [-4.12214, 6.12214]],  \\
            [[-42.63182, 44.63182], [-86.26364, 88.26364]], \\
            [[-49.56259, 51.56259], [-100.12518, 102.12518]] \\
        \end{gathered}
    \end{equation}

    \quad Решения совпали, подставим в правую часть:

    \begin{equation}
        \begin{gathered}
            Ax = [[-29.79476, 32.15777], [-60.77102, 63.13403]], \\
            [[-29.46804, 31.84923], [-60.12668, 62.50787]],  \\
            [[-29.25524, 31.64848], [-59.70709, 62.10033]], \\
            [[-29.15068, 31.54993], [-59.50098, 61.90023]] \\
        \end{gathered}
    \end{equation}
    
    \section{Обсуждение}
    \quad Можно заметить, что точечное решение лежит внутри ядер нечетких чисел.
    Результаты решения разными методами совпали, что обусловленно особенностями арифметики трапиевидных нечетких чисел. 
    
    Кроме того, можно заметить, что при подстановке полученного вектора $ x $ в изначальную систему интервальная неопределенность ядра и носителя значений вектора правой части увеличилась на 5 порядков относительно изначального $ bf $.
\end{document}