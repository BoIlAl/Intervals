\documentclass[a4paper,12pt]{article}

\usepackage[hidelinks]{hyperref}
\usepackage{amsmath}
\usepackage{mathtools}
\usepackage{shorttoc}
\usepackage{cmap}
\usepackage[T2A]{fontenc}
\usepackage[utf8]{inputenc}
\usepackage[english, russian]{babel}
\usepackage{xcolor}
\usepackage{graphicx}
\usepackage{float}
\graphicspath{{./img/}}

\definecolor{linkcolor}{HTML}{000000}
\definecolor{urlcolor}{HTML}{0085FF}
\hypersetup{pdfstartview=FitH,
            linkcolor=linkcolor,
            urlcolor=urlcolor,
            colorlinks=true}

\DeclarePairedDelimiter{\floor}{\lfloor}{\rfloor}

\renewcommand*\contentsname{Содержание}

\newcommand{\plot}[3]{
    \begin{figure}[H]
        \begin{center}
            \includegraphics[scale=0.6]{#1}
            \caption{#2}
            \label{#3}
        \end{center}
    \end{figure}
}

\begin{document}
    \include{title}
    \newpage

    \tableofcontents
    \listoffigures
    \newpage

    \section{Постановка задачи}
    \quad Имеется выборка $ (X, (Y)) $. $ X $ -- множество вещественных чисел,
    $ Y $ -- множество интервалов. Необходимо по построенному коридору совместных зависимостей провести анализ остатков.
    
    \section{Теория}
    \quad Для выборки $ (X, (Y))$, $ X = \{x_i\}_{i=1}^{n}, {Y} = \{{y}_i\}_{i=1}^{n} $
    ($ x_i $ - точеный, $ {y}_i $ - интервальный) и $ \Upsilon(x) $ -- коридора совместных зависимостей определены следующие отношения.

    Размах:
    \begin{equation}
        l(x, y) = \frac{\Upsilon(x)}{rad(y)}
    \end{equation}

    Относительный остаток:
    \begin{equation}
        r(x, y) = \frac{mid(y) - mid(\Upsilon(x))}{rad(y)}
    \end{equation}

    Границы статусов наблюдений задаются выражениями:
    \begin{equation}
        |r(x, y)| \leq 1 - l(x, y)
        \label{e:inner}
    \end{equation}

    \ref{e:inner} выполнено для внутренних наблюдений. Если достигается равенство, то наблюдение граничное.
    
    \begin{equation}
        |r(x, y)| > 1 + l(x, y)
        \label{e:remainder}
    \end{equation}

     \ref{e:remainder} выполнено для выбросов.
    
    \begin{equation}
        l(x, y) > 1
        \label{e:outter}
    \end{equation}
    
    При совместном невыполнении \ref{e:inner} и \ref{e:remainder} наблюдение
    считается внешним. Если, кроме этого, выполнено еще и \ref{e:outter}, то
    наблюдение -- абсолютно внешнее.

    \section{Реализация}
    \quad Весь код написан на языке Python (версии 3.9.5).
    \href{https://github.com/BoIlAl/Intervals/tree/master/lab3}{Ссылка на GitHub с исходным кодом}.

    \section{Результаты}
    \quad Данные были взяты из файлов \textsl{rawData/0.05V\_sp321.dat}, \textsl{rawData/-0.05V\_sp547.dat}, \textsl{rawData/0.15V\_sp9.dat}, \textsl{rawData/-0.15V\_sp831.dat}, \textsl{rawData/0.25V\_sp320.dat}, \textsl{rawData/-0.25V\_sp484.dat}, \textsl{rawData/0.35V\_sp300.dat}, \textsl{rawData/-0.35V\_sp670.dat}, \textsl{rawData/-0.45V\_sp31.dat} и \textsl{rawData/0.45V\_sp176.dat}. С коррекцией при помощи вспомогательных данных из файла \textsl{rawData/0.0V\_sp812.dat}. Набор значений
    $ X = [-0.45, -0.35, -0.25, -0.15, -0.05, 0.05, 0.15, 0.25, 0.35, 0.45] $. Набор значений $ Y_1 $ определяется как интервальная
    мода данных из соответсвующих файлов (изначальные данные обыинтерваливаются с $eps = 600 $). Набор значений $ Y_2 $ определяется как обынтерваленное среднее из соответсвующих файлов ($eps = 125 $). 
    
    Начнем с $ Y_1 $. Итоговая выборка:
    \plot{img/X, (Y1).png}{Исходная интервальная выборка $ X, (Y_1) $}{p:y1}

    Следует отметить, что 1 и 7 интервалы малы настолько, что отсутствуют на графике. Однако учитываются в дальнейших вычислениях. 
    
    Точечная линейная регрессия имеет вид: 
    \plot{img/Regression X, (Y1).png}{Точечная линейная регрессия выборки $ X, (Y_1) $}{p:regY1}

    Построим информационное множество:
    \plot{img/Inform X, (Y1).png}{Информационное множество выборки $ X, (Y_1) $}{p:infY1}

    Коридор совместных зависимостей:
    \plot{img/Corridor X, (Y1).png}{Коридор совместных зависимостей выборки $ X, (Y_1) $}{p:corY1}

    4 и 6 наблюдения лежат вне коридора совместных зависимостей.

    Строим диаграмму статусов:
    \plot{img/Status diagram X, (Y1).png}{Диаграмма статусов выборки $ X, (Y_1) $}{p:dsY1}

    Наблюдения 4, 6 являются выбросами. 1, 3, 5, 7 наблюдения граничные, остальные внутренние.

    Теперь $ Y_2 $. Итоговая выборка:
    \plot{img/X, (Y2).png}{Исходная интервальная выборка $ X, (Y_2) $}{p:y2}

    Точечная линейная регрессия имеет вид: 
    \plot{img/Regression X, (Y2).png}{Точечная линейная регрессия выборки $ X, (Y_2) $}{p:regY2}

    Построим информационное множество:
    \plot{img/Inform X, (Y2).png}{Информационное множество выборки $ X, (Y_2) $}{p:infY2}

    Коридор совместных зависимостей:
    \plot{img/Corridor X, (Y2).png}{Коридор совместных зависимостей выборки $ X, (Y_2) $}{p:corY2}

    Строим диаграмму статусов:
    \plot{img/Status diagram X, (Y2) textless.png}{Диаграмма статусов выборки $ X, (Y_2) $}{p:dsY2}

    \plot{img/Status diagram X, (Y2) zoom.png}{Диаграмма статусов выборки $ X, (Y_2) $ приближение}{p:dszY2}

    Наблюдения 1, 4, 5, 7, 10 являются граничными, остальные внутренние.
    
    \section{Обсуждение}
    \quad Из полученых результатов можно заметить, что диаграммы статусов соотносятся с построенными информационными множествами: количество наблюдений, задающих информационное множество, совпадает с количеством наблюдений со статусом "граничное". Также статусы совпадают с положением наблюдений в коридоре совместных зависимостей.
\end{document}